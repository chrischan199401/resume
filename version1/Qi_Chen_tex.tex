% LaTeX resume using res.cls
\documentclass[line,margin]{res}
\usepackage[colorlinks]{hyperref}
%\usepackage{helvetica} % uses helvetica postscript font (download helvetica.sty)
%\usepackage{newcent}   % uses new century schoolbook postscript font
\begin{document}
\name{\LARGE{Qi Chen}}
% \address used twice to have two lines of address
\address{Address: 325 W. Adams Blvd 4105, Los Angeles CA 90007 \quad \quad Email: \href{mailto:chen147@usc.edu}{\textcolor{blue}{chen147@usc.edu}}}
\address{\hspace{101mm} Phone: (213) 479-3339}

\begin{resume}
\vspace{-2mm}
\section{OBJECTIVE}
	Seeking Software Engineering Internship for Summer 2017
\vspace{-1.5mm}
\section{EDUCATION}
                {\sl \href{http://www.usc.edu/}{\textcolor{blue}{\textbf{University of Southern California}}}}  \hfill Aug.2016 - May.2018
                    \begin{itemize}
                    \item M.S. in Computer Science \quad GPA: 4.0/4.0
                    \end{itemize}
                \vspace{-2mm}
                {\sl \href{http://www.njupt.edu.cn/}{\textcolor{blue}{\textbf{Nanjing University of Posts and Telecommunications}}}}  \hfill Sep.2012 - June.2016
                 \begin{itemize}
                   \item B.E. in Electrical and Computer Engineering \quad GPA:3.8/4.0                 \end{itemize}

\vspace{-2.5mm}

\section{SKILLS}
            {\sl \textbf{Programming Languages}}: Java, Python, C++, Matlab\\
            {\sl \textbf{Web Technologies}}: HTML/CSS, Javascript, J2EE, PHP, Apache, AJAX, JSON \\
            {\sl \textbf{Other Technologies}}: SQL, Spark, Hadoop, Linux, Android SDK\\
\vspace{-2.5mm}

\section{RESEARCH \\ EXPERIENCE}
        {\sl \textbf{Energy Efficient Resource Allocation in Data Centers}} \hfill May.2014 - Mar.2015 \\[1mm]
        \vspace{0.1mm}
        \quad \textbf{Research Assistant}, Supervisor: Prof.Jianxin Chen
         \vspace{1mm}
        \begin{itemize}
            \item Proposed a \textbf{probabilistic adaptive} overload detection based on central limited theorem to trade off power cost and Service Level Agreement (SLA) cost
	\item Transformed dynamic VM consolidation into an \textbf{optimization problem}
	\item Evaluated the scheme by \textbf{CloudSim} and the results reduced about 77.5\%-82.4\% migrations and saved up to 39.3\%-42.2\% power consumption compared with First Fit Decreasing
\item \textbf{Publication:} \underline{\textbf{Qi Chen}}, Jianxin Chen, et al. "Utilization-based VM consolidation scheme for power efficiency in cloud data centers," in \emph{Communication Workshop (ICC), 2015 IEEE International Conference on}, pp.1928-1933, 8-12 June 2015APA (EI)
\item \textbf{Techniques Used:} Java, CloudSim, Heuristic Function, Optimization Search
        \end{itemize}
 %           \vspace{-1mm}
 %           \item Computational Crossword Solver
 %           \vspace{-2mm}
%                \begin{itemize}
%                    \item Used a unigram model to compute probabilities for possible answers to each clue from similar clues in a clue database
%                    \vspace{-1mm}
 %                   \item Used an iterative algorithm to fill in the crossword puzzle grid word by word.
 %               \end{itemize}
 %       {\sl \href{http://en.sjtu.edu.cn/}{\textcolor{blue}{\textbf{Shanghai Jiao Tong University}}}}, Shanghai, China  \hfill Jul.2013 - Mar.2014 \\[1mm]
 %       \vspace{0.1mm}
  %      \quad \textbf{Research Assistant}, Supervisor: {\href{http://automation.sjtu.edu.cn/en/ShowPeople.aspx?info_id=361&info_lb=326&flag=224}{\textcolor{blue}{Prof.Chengnian Long}}}
   %     \vspace{1mm}
    %    \begin{itemize}
     %       \item Human Activity Inference on Smartphone
      %      \vspace{-2mm}
       %         \begin{itemize}
        %        \item Developed a modified Apriori algorithm to mine  relationships among various activities and then to infer one activity from other already-known activities.
         %       \end{itemize}
        %\end{itemize}

 %       {\sl \textbf{A Multi-level System for Crossword Solving}}
%        \vspace{1mm}
%        \begin{itemize}
%            \item Supervisor: Prof. Yong Yu
%            \item Collaborator: Kan Ren, Yiding Tian
%            \item Previous research to solve crossword only focused on clue-answer database or Wikipedia. We designed several secondary databases, such as synonym database and slang database. Then We integrated all databases in a multi-level system. According to the type of input clue, system would search different databases in a specific order, sequentially or parallelly.
%        \end{itemize}
%
%        {\sl \textbf{Predicting Stock Market Indicators through Chinese Guba  }}
%        \vspace{1mm}
%        \begin{itemize}
%            \item Supervisor: Prof. Yong Yu
%            \item I tried to find the relation between stock market indicators and investor sentiment hid in Guba. The project contained two parts. The first part was to analyze the sentiment from comments and posts in Guba. This part(Chinese sentiment analysis) was based on a Chinese lexicon and many semantic rules. Second part focuses on how to predict stock market indicators, especially close price, through the sentiment features.
%               % The solution was an ensemble of many models. According to specific purpose of the model, each model uses different features and data sets to train. To get a better result, there were 3 underlying algorithms using in models.
%        \end{itemize}


\section{SELECTED \\ PROJECTS}
           {\sl \textbf{Congress Information Search Web and IOS APP}}           \hfill Sept.2016 - Dec.2016 \\[1mm]
             \vspace{-5mm}
             \begin{itemize}
	\item Designed a web-based information system to search congress information based on \textbf{HTML5/CSS}
%		\item Implemented all functions using \textbf{AJAX, JSON and JQuery}
		\item Developed that application to \textbf{IOS} platform
		\item \textbf{Techniques Used:} HTML5/CSS, AJAX, JSON, Bootstrap, jQuery, AWS and IOS APP
             \end{itemize}
             \vspace{-3mm}
            {\sl \textbf{Rehabilitation System Based on Wearable Computing}}\hfill Aug.2015 - Sep.2015 \\[1mm]
            \vspace{-5mm}
             \begin{itemize}
	\item Designed a three-dimensional wearable \textbf{human motion capture} module with \textbf{Kinect SDK}
             \vspace{-1mm}
	\item Applied \textbf{Extended Kalman Filter} to improve the accuracy and stability of motion tracking
	\item \textbf{Techniques Used:} Kinect SDK, C++, kalman filter
             \end{itemize}
             \vspace{-3mm}
            {\sl \textbf{Online Social Network APP on Android Platform}}\hfill Mar.2014 - July.2014 \\[1mm]
            \vspace{-5mm}
            \begin{itemize}
	\item Implemented \textbf{self-designed} User database tables based on \textbf{MySQL}
            \vspace{-1mm}
	\item Developed several online basic Social Network's functions via \textbf{J2EE}, including video chatting, social updates and commenting, etc
	\item Developed \textbf{intelligent recommender system} by users' affection, employing several \textbf{machine learning} algorithms
	\item \textbf{Techniques Used:} Java, MySQL, Android SDK, J2EE, JSON, Tomcat

            \end{itemize}
\end{resume}
\end{document}







